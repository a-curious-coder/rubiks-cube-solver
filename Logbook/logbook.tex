\documentclass[12pt]{report} % Font and size
    \usepackage{times}
\usepackage[margin=0.75in]{geometry} % Margin size
\usepackage[english]{babel} % Language
\usepackage{graphicx} % Include images with this library
\usepackage[export]{adjustbox}
\usepackage{subcaption}
\usepackage{wrapfig}
\usepackage{array}
\usepackage{}
\usepackage{multirow}

\begin{document}
% Creating title page
    % Title
    \title{
        \centerline{\includegraphics[width = 25mm]{Uni.png}}
        \vspace{1.5cm}
        \textsc{\LARGE University of Reading}\\[0.25cm]
        \textsc{\Large Department of Computer Science}\\[1cm]
        \textsc{\large Computer Science Undergraduate - }
        \textsc{\large Rubik's Cube Solver Logbook}\\[0.75cm]
    }
        % Author
    \author{
        \large
        \textsc{Callum Claton David McLennan}\\[0.25cm]
        \textsc{\normalsize \textit{Supervisor:} Martin Lester}\\[1cm]
    }
    % End of title
    \maketitle
\centering
\section*{Rubik's Cube Solver - Logbook}
\begin{table}[!ht]
    \centering
    \begin{tabular}{
        |>{\centering\arraybackslash}p{0.1\linewidth}
        |p{0.8\linewidth}|}
    \hline
    \textbf{Date} & \textbf{Description} \\ \hline
    % Entry 1
    10/08/2020   & 
    Created project. Followed tutorial to create emulator which allowed me to implement moves to the cube, animations when the moves occured and a scramble function but it doesn't work yet.            
    \\ \hline
    % Entry 2
    12/08/2020   &
    Refind scrambling function, the cube now successfully scrambles with random moves.
    \\ \hline
    % Entry 3
    14/08/2020  &
    Reverse the Rubik's cube scramble to give the illusion that it's solved (Was apart of the tutorial). Updated the README with what some gifs and research information.
    \\ \hline
    % Entry 4
    16/08/2020  &
    Created multiple cube sizes. The only limitation on size is the computing power available. The program generates cubies within the cube which is unecessarily using up computing power. Moves/scrambling is only catered for 3x3x3 cube.
    \\ \hline
    % Entry 5
    17/08/2020  &
    Removed inner cubes,   \newline
    Set face colours to only the visible faces,    \newline
    Added FPS count,   \newline
    Added cube size,   \newline
    Added speed control,   \newline
    Fixed even numbered cube scramble.   \newline
    Catered moves to work for any sized cube
    Working on identifying when colours on neighbouring cubies faces match.
    I will then work on a human algorithm to solve the cube.
    After, I will begin working on a computer algorithm to solve the cube
    Besides these objectives, I'll be refining and adapting code to be easier to read and fit needs
    \\ \hline
    % Entry 6
    30/08/2020  &
    In process of adding X, Y, Z whole cube rotations.
    Algorithm files have been made, \newline
    In progress with creating X,Y,Z rotations for ease of solve, \newline
    Issues with setting cubies to correct positions
    Cubies change colours but they need to change positions on cube.
    \\ \hline
    % Entry 7
    04/09/2020  &
    Finished new turning functions  \newline
    Updated README. \newline
    Fixed a few rotation/moving bugs.   \newline
    Created a few debugging print functions.    \newline
    Turning functions do not currently cater for any cube size that is not 3x3x3    \newline

    \textbf{TODO}:
    Cater turning functions for cubes of any size, \newline
    Adapt code to correctly act on double moves,    \newline
    Fix reverse scramble function,  \newline
    Clean code up for better understanding, \newline
    Remove X,Y,Z rotations from counting as scramble/solve ,, \newline moves since they don't modify the cube's scramble state
    Start working on human solving algorithm after these jobs are complete.
    \\ \hline
    \end{tabular}
\end{table}

\begin{table}[!ht]
    \centering
    \begin{tabular}{
        |>{\centering\arraybackslash}p{0.1\linewidth}
        |p{0.8\linewidth}|}
    \hline
    \textbf{Date} & \textbf{Description} \\ \hline
    % Entry 8
    09/09/2020  &
    Catered the turning functions for all cube sizes. \newline
    Added more control over cubies  \newline
    Refined controls over speed \newline
    Can switch between cube sizes during runtime    \newline
    Cleaned up some code    \newline
    \textbf{TODO}:
    Fix cubie colour change problem that occurs for unknown reason  \newline
    Clean up code to be more conventional. Upon further research, I've realised I probably went a bit overkill with the comments
    \\ \hline
    % Entry 9
    12/09/2020  &
    Updates to human algorithm function - Solves white cross on cube.   \newline
    Currently working on additional steps of the human algorithm.
    \newline
    Although it's not a mandatory requirement, I feel it will heavily familiarise me with every aspect of the cube - ready for more complex algorithms.
    \newline
    Added some boolean operators for when the program should call the solving function
    \newline
    Also when to hide the HUD - gives the program a cleaner look
    \newline
    Issues\newline
    - Going to disregard issues for higher cubes until I've finished the human algorithm catering for a 3x3x3
    (I'm still unsure the reason for random colour changes for bigger cubes as of yet.)\newline
    \textbf{TODO}:\newline
    - Add 2D visualisation of the cube\newline
        - Allow user to custom scramble the cube by clicking 2D visualisation\newline
    - Research and discover methods of speeding up the programs performance as cube's above 25x25x25 puts FPS below 30.\newline
    \\ \hline
    13/09/2020  &   
    Finished stage 2 of human algorithm \newline
    Successfully arranges white cross on cube   \newline
    Successfully positions corners on bottom of cube    \newline
    - Changed README banner
    \\ \hline
    15/09/2020  &
    Completed step 3 (F2L) on human algorithm   \newline
    - Discovered looping issue with step 2 - now fixed. \newline
    - Still need to clean up code after all steps are completed for algorithm   \newline
    - Progressing onto step 4   \newline
    - Updated README
    \\ \hline
    17/09/2020  &
    Finished human algorithm steps 3,4,5,6 and 7.   \newline
    Successfully solves the Rubik's cube.   \newline
    Prints most of the solving moves to HUD \newline
    Need to refine code to make easier to understand.   \newline
    Need to clean code up   \newline
    Planning on adding 2D view of the Rubik's RubiksCube    \newline
    Going to do some research of local search algorithm \newline
    Will postpone catering for bigger cubes until further notice.   \newline
    \\ \hline
    19/09/2020  &
    Added 2D visualisation of cube \newline
    Hoping to allow the user to modify the cubie's colours via the 2D visualisation in the near future
    Cleaned up code (re-evaluated comments)
    \\ \hline
    \end{tabular}
\end{table}

\begin{table}
    \centering
    \begin{tabular}{
        |>{\centering\arraybackslash}p{0.1\linewidth}
        |p{0.8\linewidth}|}
    \hline
    \textbf{Date} & \textbf{Description} \\ \hline
    28/10/2020  &
    On the hunt for what's causing the cube to randomly change scramble states. I suspect it could be applying the "solution" it comes up with straight away then applying it again with animations? I'm unsure as of yet as further research is required.\newline
    I'm also working on trying to clean up the code to make it more logical and easy to follow.
    \\ \hline
    \end{tabular}
\end{table}
\end{document}