\documentclass[12pt]{report} % Font and size
    \usepackage{times}
\usepackage[margin=0.75in]{geometry} % Margin size
\usepackage[english]{babel} % Language
\usepackage{graphicx} % Include images with this library
\usepackage[export]{adjustbox}
\usepackage{subcaption}
\usepackage{wrapfig}
\usepackage{array}
\usepackage{multicol}
\usepackage{glossaries}
\setlength{\columnsep}{1cm}
\usepackage{biblatex}
\addbibresource{sample.bib}
\renewcommand{\thesection}{\arabic{section}}
% Requirements
% 10,000 - 20,000 words

% Creating title page
    % Title
\title{
    \centerline{\includegraphics[width = 25mm]{Images/Uni.png}}
    \vspace{1.5cm}
    \textsc{\LARGE University of Reading}\\[0.25cm]
    \textsc{\Large Department of Computer Science}\\[1cm]
    \textsc{\large Computer Science Undergraduate Report - }
    \textsc{\large Rubik's Cube Solver}\\[0.75cm]
}
    % Author
\author{
    \large
    \textsc{Callum Claton David McLennan}\\[0.25cm]
    \textsc{\normalsize \textit{Supervisor:} Martin Lester}\\[1cm]
    A report submitted in partial fulfilment of the requirements of\\the University of Reading for the degree of\\
    Bachelor of Science in \textit{Computer Science}
    \date{\today}
}
% End of title

\begin{document}
\pagenumbering{roman}
\maketitle
\newpage
% Declaration
\section*{Declaration}
\vspace{1cm}
I, Callum, Claton, David, McLennan, of the Department of Computer Science, University of Reading, confirm that all the sentences, figures, tables, equations, code snippets, artworks, and illustrations in this report are original and have not been taken from any other person’s work, except where the works of others have been explicitly acknowledged, quoted, and referenced. I understand that if failing to do so will be considered a case of plagiarism. Plagiarism is a form of academic misconduct and will be penalised accordingly.
\vspace{1cm}
\begin{flushright}Callum David Claton McLennan \\ \date{\today}\end{flushright}
\newpage
% Abstract
\section*{Abstract}
random words that attract the reader
% Guidance on abstract writing: An abstract is a summary of a report in a single paragraph up to a maximum of 250 words. An abstract should be self-contained, and it should not refer to sections, figures, tables, equations, or references. An abstract typically consists of sentences describing the following four parts: (1) introduction (background and purpose of the project), (2) methods, (3) results and analysis, and (4) conclusions. The distribution of these four parts of the abstract should reflect the relative proportion of these parts in the report itself. An abstract starts with a few sentences describing the project’s general field, comprehensive background and context, the main purpose of the project; and the problem statement. A few sentences describe the methods, experiments, and implementation of the project. A few sentences describe the main results achieved and their significance. The final part of the abstract describes the conclusions and the results implications to the relevant field.

% Acknowledgements section - optional
\newpage
\section*{Glossary}
\subsection*{Terminology}
\begin{table}[!ht]
\centering
\begin{tabular}{|l|l|}
\hline
Cubie                & One of many smaller cubes that make up the Rubik's cube.                                   \\ \hline
Center               & A cubie with one colour on the face in the center of the cube.                             \\ \hline
Edge                 & An edge cubie has two colours as they're on the edge of the cube.                          \\ \hline
Corner               & A corner cubie has 3 colours and there are always 8, regardless of cube size.              \\ \hline
Face                 & A face is a side of a Rubik's Cube. There are 6 faces regardless of size.                  \\ \hline
\multicolumn{2}{|l|}{A letter by itself refers to a clockwise rotation of a single face by 90°.}                  \\ \hline
\multicolumn{2}{|l|}{A letter with a ' ' ' is a 'prime move' which means the face rotates counter-clockwise 90°.} \\ \hline
\multicolumn{2}{|l|}{A letter with the number 2 after it marks a double turn 180°.}                               \\ \hline
\multicolumn{2}{|l|}{X, Y, Z rotations aren't normally required to solve a cube. These are whole cube rotations.} \\ \hline
\end{tabular}
\end{table}

\subsection*{Moves}
\begin{table}[!ht]
\centering
\begin{tabular}{|c|c|c|c|c|c|c|c|c|c|}
\hline
             & Front & Right & Up & Left & Back & Down & \multicolumn{3}{c|}{Entire cube rotation} \\ \hline
Normal moves & F     & R     & U  & L    & B    & D    & X            & Y            & Z           \\ \hline
Prime moves  & F'    & R'    & U' & L'   & B'   & D'   & X'           & Y'           & Z'          \\ \hline
Double moves & F2    & R2    & U2 & L2   & B2   & D2   & \multicolumn{3}{c|}{N/A}                     \\ \hline
\end{tabular}
\end{table}
% Contents
\tableofcontents
% \listoffigures
% \listoftables
\newpage
% ---------------------------------------------------------------------------
% Report begins
% ---------------------------------------------------------------------------
\pagenumbering{arabic}
\setcounter{page}{1}
% Introduction
\section{Introduction}
This is my attempt at a Rubik's Cube Solver. I will be doing my best to log the progress I make and what motivate my decisions regarding  features and methods I include.
\subsection{Background}
The Rubik's Cube is a well known puzzle all around the world and it's considered extremely difficult to solve.
\subsection{Aims and Objectives}
My ambitious goal is to create some form of heuristic based search algorithm or perhaps a constraint solver to solve the cube and perhaps various cube sizes. However, it's very likely I could stray and use a different, more efficient and well suited algorithm to solve the cube(s).
\\
A cool feature I'm eager to add is a custom scramble where the user can manually change the digital cube to their real world physical cube's scrambled state then get my program to solve it and provide the algorithm (steps) to solving that particular scramble.
\\
% Needs more info for this section


\subsection{Research Hypothesis}
Although this project is ambitious, I'm hoping that with my current experience solving Rubik's Cubes, this project will move forward more fluently than previous projects I've pursued and hopefully I'm able to better diagnose, understand and solve issues during the project creation.
\newline
\newline
I'm excited to discover if I can create or piece together an algorithm that is versatile enough to solve different sized cubes but considering that I feel this project is a big challenge regarding just the 3x3x3 sized cube, I wouldn't be too disappointed if I didn't succeed in creating an algorithm that can solve all sized cubes.
\newpage
% Literature review
\section{Literature review}

\newpage
% Methodology
\section{Methodology}
% Method used
%   How does it work?
%      Storage method?
    
%   Measurements
%       Computing power/time?
%       Moves needed to solve
\subsection{Human algorithm}

\subsection{Search algorithm}
\subsubsection{Layer by layer approach}
\subsubsection{Kociemba approach}

\newpage
% Results and analysis
\section{Results and analysis}

\newpage
% Conclusions and future work
\section{Conclusions and future work}

\printbibliography
\end{document}
